%
% 6.006 problem set 1
%
\documentclass[12pt,twoside]{article}

\usepackage{amsmath}

\input{macros}

\setlength{\oddsidemargin}{0pt}
\setlength{\evensidemargin}{0pt}
\setlength{\textwidth}{6.5in}
\setlength{\topmargin}{0in}
\setlength{\textheight}{8.5in}

% Fill these in!
\newcommand{\theproblemsetnum}{1}
\newcommand{\releasedate}{September 8, 2011}
\newcommand{\partaduedate}{Thursday, September 15}
\newcommand{\tabUnit}{3ex}
\newcommand{\tabT}{\hspace*{\tabUnit}}

\begin{document}

\handout{Problem Set \theproblemsetnum}{\releasedate}

\newif\ifsolution
\solutiontrue
\newcommand{\solution}{\textbf{Your Solution:}}

\textbf{Both theory and programming questions} are due {\bf \partaduedate} at {\bf 11:59PM}.
%
Please download the .zip archive for this problem set, and refer to the
\texttt{README.txt} file for instructions on preparing your solutions.
%
Remember, your goal is to communicate. Full credit will be given only
to a correct solution which is described clearly. Convoluted and
obtuse descriptions might receive low marks, even when they are
correct. Also, aim for concise solutions, as it will save you time
spent on write-ups, and also help you conceptualize the key idea of
the problem.

We will provide the solutions to the problem set 10 hours after the problem set
is due, which you will use to find any errors in the proof that you submitted.
You will need to submit a critique of your solutions by \textbf{Tuesday,
September 20th, 11:59PM}. Your grade will be based on both your solutions and
your critique of the solutions.

\setlength{\parindent}{0pt}

\medskip

\hrulefill

\textbf{Collaborators:}
%%% COLLABORATORS START %%%
None.
%%% COLLABORATORS END %%%

\begin{problems}

\problem \points{15} \textbf{Asymptotic Practice}

For each group of functions, sort the functions in increasing order of
asymptotic (big-O) complexity:

\begin{problemparts}

\problempart \points{5} \textbf{Group 1:}

$$
\begin{array}{rcl}
f_1(n) &=& n^{0.999999} \log n \\
f_2(n) &=& 10000000 n \\
f_3(n) &=& 1.000001^n \\
f_4(n) &=& n^2
\end{array}
$$

\ifsolution \solution{}
%%% PROBLEM 1(a) SOLUTION START %%%
1, 2, 4, 3
%%% PROBLEM 1(a) SOLUTION END %%%
\fi

\problempart \points{5} \textbf{Group 2:}

$$
\begin{array}{rcl}
f_1(n) &=& 2^{2^{1000000}} \\
f_2(n) &=& 2^{100000n} \\
f_3(n) &=& \displaystyle \binom{n}{2} \\
f_4(n) &=& n \sqrt{n}
\end{array}
$$

\ifsolution \solution{}
%%% PROBLEM 1(b) SOLUTION START %%%
1, 4, 3, 2
%%% PROBLEM 1(b) SOLUTION END %%%
\fi

\problempart \points{5} \textbf{Group 3:}

$$
\begin{array}{rcl}
f_1(n) &=& n^{\sqrt{n}} \\
f_2(n) &=& 2^n \\
f_3(n) &=& n^{10} \cdot 2^{n / 2} \\
f_4(n) &=& \displaystyle\sum_{i = 1}^{n} (i + 1)
\end{array}
$$

\ifsolution \solution{}
%%% PROBLEM 1(c) SOLUTION START %%%
4, 3, 1, 2
%%% PROBLEM 1(c) SOLUTION END %%%
\fi

\end{problemparts}

\problem \points{15} \textbf{Recurrence Relation Resolution}

For each of the following recurrence relations,
pick the correct asymptotic runtime:

\begin{problemparts}

\problempart \points{5}
Select the correct asymptotic complexity
of an algorithm with runtime $T(n, n)$
where 
$$
\begin{array}{rcll}
T(x, c) &=& \Theta(x) & \textrm{ for $c \le 2$}, \\
T(c, y) &=& \Theta(y) & \textrm{ for $c \le 2$, and} \\
T(x, y) &=& \Theta(x + y) + T(x / 2, y / 2).
\end{array}
$$

\begin{enumerate}
\item $\Theta(\log n)$.
\item $\Theta(n)$.
\item $\Theta(n \log n)$.
\item $\Theta(n \log^2 n)$.
\item $\Theta(n^2)$.
\item $\Theta(2^n)$.
\end{enumerate}

\ifsolution \solution{}
%%% PROBLEM 2(a) SOLUTION START %%%
2
%%% PROBLEM 2(a) SOLUTION END %%%
\fi

\problempart \points{5}
Select the correct asymptotic complexity
of an algorithm with runtime $T(n, n)$
where 
$$
\begin{array}{rcll}
T(x, c) &=& \Theta(x) & \textrm{ for $c \le 2$}, \\
T(c, y) &=& \Theta(y) & \textrm{ for $c \le 2$, and} \\
T(x, y) &=& \Theta(x) + T(x, y / 2).
\end{array}
$$

\begin{enumerate}
\item $\Theta(\log n)$.
\item $\Theta(n)$.
\item $\Theta(n \log n)$.
\item $\Theta(n \log^2 n)$.
\item $\Theta(n^2)$.
\item $\Theta(2^n)$.
\end{enumerate}

\ifsolution \solution{}
%%% PROBLEM 2(b) SOLUTION START %%%
3
%%% PROBLEM 2(b) SOLUTION END %%%
\fi

\problempart \points{5}
Select the correct asymptotic complexity
of an algorithm with runtime $T(n, n)$
where 
$$
\begin{array}{rcll}
T(x, c) &=& \Theta(x) & \textrm{ for $c \le 2$}, \\
T(x, y) &=& \Theta(x) + S(x, y / 2), \\
S(c, y) &=& \Theta(y) & \textrm{ for $c \le 2$, and} \\
S(x, y) &=& \Theta(y) + T(x / 2, y).
\end{array}
$$

\begin{enumerate}
\item $\Theta(\log n)$.
\item $\Theta(n)$.
\item $\Theta(n \log n)$.
\item $\Theta(n \log^2 n)$.
\item $\Theta(n^2)$.
\item $\Theta(2^n)$.
\end{enumerate}

\ifsolution \solution{}
%%% PROBLEM 2(c) SOLUTION START %%%
2
%%% PROBLEM 2(c) SOLUTION END %%%
\fi

\end{problemparts}

\section*{Peak-Finding}

In Lecture 1,
you saw the peak-finding problem.
As a reminder,
a \emph{peak} in a matrix
is a location with the property that its four neighbors
(north, south, east, and west)
have value less than or equal to the value of the peak.
We have posted Python code for solving this problem
to the website in a file called \texttt{ps1.zip}.
In the file \texttt{algorithms.py},
there are four different algorithms
which have been written
to solve the peak-finding problem,
only some of which are correct.
Your goal is to figure out
which of these algorithms are correct
and which are efficient.

\problem \points{16} \textbf{Peak-Finding Correctness}

\begin{problemparts}

\problempart \points{4} Is \texttt{algorithm1} correct?
\begin{enumerate}
\item Yes.
\item No.
\end{enumerate}

\ifsolution \solution{}
%%% PROBLEM 3(a) SOLUTION START %%%
1
%%% PROBLEM 3(a) SOLUTION END %%%
\fi

\problempart \points{4} Is \texttt{algorithm2} correct?
\begin{enumerate}
\item Yes.
\item No.
\end{enumerate}

\ifsolution \solution{}
%%% PROBLEM 3(b) SOLUTION START %%%
1
%%% PROBLEM 3(b) SOLUTION END %%%
\fi

\problempart \points{4} Is \texttt{algorithm3} correct?
\begin{enumerate}
\item Yes.
\item No.
\end{enumerate}

\ifsolution \solution{}
%%% PROBLEM 3(c) SOLUTION START %%%
2
%%% PROBLEM 3(c) SOLUTION END %%%
\fi

\problempart \points{4} Is \texttt{algorithm4} correct?
\begin{enumerate}
\item Yes.
\item No.
\end{enumerate}

\ifsolution \solution{}
%%% PROBLEM 3(d) SOLUTION START %%%
1
%%% PROBLEM 3(d) SOLUTION END %%%
\fi

\end{problemparts}

\problem \points{16} \textbf{Peak-Finding Efficiency}

\begin{problemparts}

\problempart \points{4} What is the worst-case runtime of \texttt{algorithm1} on a problem of size $n \times n$?
\begin{enumerate}
\item $\Theta(\log n)$.
\item $\Theta(n)$.
\item $\Theta(n \log n)$.
\item $\Theta(n \log^2 n)$.
\item $\Theta(n^2)$.
\item $\Theta(2^n)$.
\end{enumerate}

\ifsolution \solution{}
%%% PROBLEM 4(a) SOLUTION START %%%
3
%%% PROBLEM 4(a) SOLUTION END %%%
\fi

\problempart \points{4} What is the worst-case runtime of \texttt{algorithm2} on a problem of size $n \times n$?
\begin{enumerate}
\item $\Theta(\log n)$.
\item $\Theta(n)$.
\item $\Theta(n \log n)$.
\item $\Theta(n \log^2 n)$.
\item $\Theta(n^2)$.
\item $\Theta(2^n)$.
\end{enumerate}

\ifsolution \solution{}
%%% PROBLEM 4(b) SOLUTION START %%%
5
%%% PROBLEM 4(b) SOLUTION END %%%
\fi

\problempart \points{4} What is the worst-case runtime of \texttt{algorithm3} on a problem of size $n \times n$?
\begin{enumerate}
\item $\Theta(\log n)$.
\item $\Theta(n)$.
\item $\Theta(n \log n)$.
\item $\Theta(n \log^2 n)$.
\item $\Theta(n^2)$.
\item $\Theta(2^n)$.
\end{enumerate}

\ifsolution \solution{}
%%% PROBLEM 4(c) SOLUTION START %%%
3
%%% PROBLEM 4(c) SOLUTION END %%%
\fi

\problempart \points{4} What is the worst-case runtime of \texttt{algorithm4} on a problem of size $n \times n$?
\begin{enumerate}
\item $\Theta(\log n)$.
\item $\Theta(n)$.
\item $\Theta(n \log n)$.
\item $\Theta(n \log^2 n)$.
\item $\Theta(n^2)$.
\item $\Theta(2^n)$.
\end{enumerate}

\ifsolution \solution{}
%%% PROBLEM 4(d) SOLUTION START %%%
2
%%% PROBLEM 4(d) SOLUTION END %%%
\fi

\end{problemparts}

\problem \points{19} \textbf{Peak-Finding Proof}

Please modify the proof below to construct a proof of correctness
for the \emph{most efficient correct algorithm}
among \texttt{algorithm2}, \texttt{algorithm3}, and \texttt{algorithm4}.

%%%The following is the proof of correctness
%%%for \texttt{algorithm4},
%%%which was sketched in Lecture 1.

\begin{quote}
We wish to show that \texttt{algorithm4}
will always return a peak,
as long as the problem is not empty.
To that end,
we wish to prove the following two statements:

{\bf 1. If the peak problem is not empty,
then \texttt{algorithm4} will always return a location.}
Say that we start with a problem of size $m \times n$.
The recursive subproblem (after row-split) examined by \texttt{algorithm4}
will have dimensions, ($m^* \times n$)
where $m^*$ is
$\lfloor m / 2 \rfloor$ or 
$\left(m - \lfloor m / 2 \rfloor - 1 \right)$.
At next step (after column-split), the recursive subproblem examined by \texttt{algorithm4} will have dimensions, $m^* \times n^*$
where $n^*$ is 
$\lfloor n^* / 2 \rfloor$ or
$\left(n - \lfloor n / 2 \rfloor - 1 \right)$
Therefore, the number of rows or number of columns in the problem strictly decreases with each recursive call as long as $m > 0$ and $n > 0$
Therefore, the number of columns in the problem
strictly decreases with each recursive call
as long as $n > 0$.
So \texttt{algorithm4} either returns a location at some point,
or eventually examines a subproblem with number of columns and rows both positive.
The only way for the number of columns or number of rows to become strictly negative,
according to the formulas that determine the size of the subproblem,
is to have $n = 0$ or $m = 0$ at some point.
So if \texttt{algorithm4} doesn't return a location,
it must eventually examine an empty subproblem.

We wish to show that there is no way that this can occur.
Assume, to the contrary,
that \texttt{algorithm4} does examine an empty subproblem.
Just prior to this,
it must examine a subproblem of size
$m \times 1$ or $m \times 2$ or $1 \times n$ or $2 \times n$.
If the problem is of size $m \times 1$,
then calculating the maximum of the central column
is equivalent to calculating the maximum of the entire problem.
Hence, the maximum that the algorithm finds must be a peak,
and it will halt and return the location.
If the problem has dimensions $m \times 2$,
then there are two possibilities:
either the maximum of the central column is a peak
(in which case the algorithm will halt and return the location),
or it has a strictly better neighbor in the other column
(in which case the algorithm will recurse
on the non-empty subproblem with dimensions $m \times 1$,
thus reducing to the previous case).
So \texttt{algorithm4} can never recurse into an empty subproblem,
and therefore \texttt{algorithm4} must eventually return a location.

{\bf 2. If \texttt{algorithm4} returns a location,
it will be a peak in the original problem.}
If \texttt{algorithm4} returns a location $(r_1, c_1)$, then $val(r_1, c_1)$ must be a peak within some recursive subproblem of size $(m^* \times n^*)$ and $val(r_1, c_1)$ must be the best value in either row $r_1$ or column $c_1$ with row and column size $m^*$ and $n^*$ respectively.
On the contrary, assume that $val(r_1, c_1)$ is not a peak in the original problem. Then at some point as location $(r_1, c_1)$ passed up the chain of recursive calls , some neighbor $(r_n, c_n)$ to $(r_1, c_1)$ must have value greater then it,i.e. 
$val(r_n, c_n) > val(r_1, c_1)$
Since ,value passed by algorthim is $val(r_1, c_1)$. This implies that value is the bestSeen must be less than returned value. i.e.
$val(r_b, c_b) < val(r_1, c_1)$.
Also $val(r_b, c_b) > val(r_n, c_n)$ , because if that neighbor had value greater than that stored in bestSeen , that neighbour's value was stored in bestSeen. 

$\implies val(r_n, c_n) < val(r_b, c_b) < val(r_1, c_1)$

This results in contradiction. Hence $val(r_1, c_1)$ is a peak in the original problem.


\end{quote}


\problem \points{19} \textbf{Peak-Finding Counterexamples}

For each incorrect algorithm,
upload a Python file giving a counterexample
(i.e. a matrix for which the algorithm returns a location
that is not a peak).

\ifsolution \solution{}
%%% PROBLEM 6 SOLUTION START %%%
\begin{verbatim}

problemMatrix =[
	[118, 4,   3,   45,  183,  145, 182, 78,  64,  151],
	[32,  158, 3,   78,  158,  60,  60,  98,  69,  50],
	[1,   4,   10,  14,  14,   152, 41,  101, 76,  194],
	[16,  14,  1,   6,   7,    183, 88,  29,  5,   18],
	[95,  5,   17,  3,   1340, 180, 163, 29,  87,  88],
	[74,  36,  19,  38,  184,  66,  24,  65,  119, 39],
	[32,  109, 107, 189, 33,   129, 34,  105, 146, 156],
	[75,  148, 83,  155, 152,  29,  66,  40,  136, 72],
	[136, 146, 75,  3,   185,  82,  7,   139, 29,  181],
	[164, 178, 0,   11,  9,    54,  19,  112, 183, 28]
]


\end{verbatim}
%%% PROBLEM 6 SOLUTION END %%%
\fi

\end{problems}

\end{document}
